\Large \textbf{درست نادرست}
\large \textbf{(۱۰ نمره)}

\normalsize \vspace{0.5cm}

\begin{enumerate}[label=(\alph*)]
	\item
	مشابه بازی قمارباز معمولی داریم:
	$$
	\begin{array}{c}
		{a_k} = p{a_{k + 1}} + q{a_{k - 1}} + r{a_k}\\
		p{a_{k + 1}} - \left( {1 - r} \right){a_k} + q{a_{k - 1}} = 0\\
		p{\lambda ^2} - \left( {1 - r} \right)\lambda  + q = 0
	\end{array}
	$$
	
	$$
	{\lambda _{1,2}} = \frac{{\left( {1 - r} \right) \pm \sqrt {{{\left( {1 - r} \right)}^2} - 4pq} }}{{2p}}
	$$
	
	$$
	\begin{array}{c}
		p + q + r = 1\\
		p + q = 1 - r\\
		{\left( {p + q} \right)^2} = {\left( {1 - r} \right)^2}\\
		p + q - 2qp = {\left( {p - q} \right)^2} = {\left( {1 - r} \right)^2} - 4qp
	\end{array}
	$$
	
	$$
	{\lambda _{1,2}} = \frac{{\left( {1 - r} \right) \pm \left| {p - q} \right|}}{{2p}} = \frac{{p + q \pm \left| {p - q} \right|}}{{2p}}
	$$
	
	بنابراین جواب های بدست آمده دقیقا برابر حالت معمولی مسئله قمارباز شد و مقادیر
	${a_k}$
	برای حالتی که
	$p \ne q$
	برابر
	$$
	{a_k} = \frac{{{{\left( {\frac{q}{p}} \right)}^N} - {{\left( {\frac{q}{p}} \right)}^k}}}{{{{\left( {\frac{q}{p}} \right)}^N} - 1}}
	$$
	
	می‌باشد.
	
	\item
	
	مشابه بخش قبل داریم:
	
	$$
	\begin{array}{c}
		{\tau _k} = p\left( {1 + {\tau _{k + 1}}} \right) + q\left( {1 + {\tau _{k - 1}}} \right) + r\left( {1 + {\tau _k}} \right)\\
		p{\tau _{k + 1}} - \left( {1 - r} \right){\tau _k} + q{\tau _{k - 1}} =  - 1
	\end{array}
	$$
	
	طبق مشابهت رابطه بالا با بخش الف،‌ جواب عمومی
	${\tau _k}$
	مشابه جواب بخش الف خواهد بود. برای جواب خصوصی با در نظر گرفتن
	${\tau _k} = ck$
	داریم:
	$$
	\begin{array}{c}
		pc\left( {k + 1} \right) - \left( {1 - r} \right)ck + qc\left( {k - 1} \right) =  - 1\\
		c\left( {pk + p - k + rk + qk - q} \right) =  - 1\\
		c = \frac{1}{{q - p}}
	\end{array}
	$$
	
	همانطور که مشهود است،‌ جواب های مقدار متوسط زمان پایان بازی نیز مشابه حالت معمولی بازی شد.
	
	\item
	با اثبات مشابه بودن این حالت به حالت معمولی،‌ جواب های این بخش نیز برابر جوابهای مسئله معمولی قمارباز می‌شود.
	(نوشتن مراحل برای اخد نمره نیاز است.)
	
	به صورت شهودی نیز مشخص بود که با چنین تغییراتی در صورتی که نسبت q به p ثابت باقی بماند مقدار
	${a_k}$ 
	تغییر نخواهد کرد. چون تغییری که در حالت فعلی ایجاد میشود تنها توسط p و q اتفاق می افتد. همچنین با افرایش r مقادیر p و q کمتر از حالتی می شود که
	$r= = 0$
	باشد و انتظار میرود زمان جذب بیشتر شود که در رابطه متناظر خود را در بخش اول که تفاضل p و q می باشد نشان می‌دهد.
	
\end{enumerate}
