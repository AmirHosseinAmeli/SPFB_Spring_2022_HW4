\Large \textbf{ویژگی‌های اتوکورلیشن فرآیند WSS}
\large \textbf{(14 نمره)}

\normalsize \vspace{0.5cm}
براساس فرآیند تصادفی WSS در دامنه حقیقی به سوالات زیر پاسخ دهید.
\begin{enumerate}[label=(\alph*)]
	\item
اثبات کنید که که اتوکورلیشن این فرآیند زوج است.
\\
در فضای حقیقی می‌دانیم که اتوکورلیشن به شکل زیر تعریف می‌شود:
$$
R_X(t - s) = R_{XX}(t, s) = \mathbb{E}[X(t)X(s)]
$$
حال براساس این تعریف زوج بودن این تابع بر اساس خاصیت جابه‌جایی ضرب، قابل اثبات است یعنی:
\begin{align*}
	R_X(t - s) &= R_{XX}(t, s) \\
	&= \mathbb{E}[X(t)X(s)]\\
	&= \mathbb{E}[X(s)X(t)]\\
	&= R_{XX}(s, t)\\
	&= R_X(s - t)
\end{align*}
	\item
اثبات کنید که تابع اتوکورلیشن دارای یک ماکزیمم در نقطه صفر است. یعنی به عبارت ریاضی:
$$
\forall \tau : R_{X}(0) \geqslant R_X(\tau)
$$
(راهنمایی: ابتدا برای اتوکورلیشن یک کران بالا بر اساس نامساوی کوشی-شوارتز پیدا کنید سپس نشان دهید که 
$ R_X(0) $
برابر با این کران است.)
\\
براساس نامساوی کوشی شوارتز در احتمالات می‌دانیم که:
$$
|\mathbb{E}[YZ]|^2 \leqslant \mathbb{E}[Y^2]\mathbb{E}[Z^2]
$$
اگر این عبارت را برای فرآیند تصادفی X(t) بنویسیم به عبارت زیر می‌رسیم.
$$
|\mathbb{E}[X(t)X(s)]|^2 \leqslant \mathbb{E}[X^2(t)]\mathbb{E}[X^2(s)]
$$
دقت شود که امید ریاضی مرتبه دوم در فرآیند WSS مستقل از زمان می‌باشد و داریم:
$$
\mathbb{E}[X^2(t)] = \mathbb{E}[X^2(s)] = R_{X} (s - s) =  R_{X} (0) 
$$
پس با این اوصاف می‌توان نامساوی را به شکل زیر بازنویسی کرد:
$$
|R_X(t-s)|^2 \leqslant R_X^2 (0)
$$
با جذر گرفتن از دو طرف و جایگذاری 
$ \tau = t - s $
به حکم مساله می‌رسیم یعنی:
$$
R_X(\tau) \leqslant R_X (0)
$$
	\item 
بر اساس این دو خاصیت با ذکر دلیل بیان کنید که کدام یک از توابع نمی‌تواند تابع اتوکورلیشن یک فرآیند WSS باشند.
$$
R_X(\tau) = exp(0.01 \times |\tau|) + 4
$$
این تابع به علت داشتن نقاطی که بزرگتر از 
$ R_X(0)  $
هستند نمی‌تواند متعلق به فرآیند WSS باشد.
$$
R_X(\tau) = exp(-|\tau - 4|) + 9 
$$
این تابع زوج نیست لذا نمی‌تواند متعلق به فرآیند WSS باشد.
$$
R_X(\tau) = exp(-|\tau| + 4) - 25 
$$
این تابع بدون مشکل است.

\end{enumerate}


