\Large \textbf{فرایند تولد و مرگ}
\large \textbf{(20 نمره)}

\normalsize \vspace{0.5cm}

\begin{enumerate}[label=(\alph*)]
	\item
	ماتریس گذار این فرایند به صورت زیر است:
	$$
	P = \left[ {\begin{array}{*{20}{c}}
			1&0&0& \cdots &0\\
			{{d_1}}&{1 - {d_1} - {b_1}}&{{b_1}}& \cdots &0\\
			0&{{d_2}}&{1 - {d_2} - {b_2}}&{{b_2}}&0\\
			\vdots & \vdots & \vdots & \ddots &0\\
			0&0&0&{{d_N}}&{1 - {d_N}}
	\end{array}} \right]
	$$
	
	حال سطری دلخواه از ماتریس
	${P^n}$
	را در نظر بگیرید:
	$$
	{P^n} = \left[ {\begin{array}{*{20}{c}}
			{{r_0}}&{{r_1}}& \cdots &{{r_{N - 1}}}&{{r_N}}
	\end{array}} \right]
	$$
	
	حال به همین سطر در ماتریس
	${P^{n + 1}}$
	نگاه بکنیم داریم:
	$$
	{P^{n + 1}} = \left[ {\begin{array}{*{20}{c}}
			{{r_0} + {d_1}{r_1}}&{\left( {1 - {d_1} - {b_1}} \right){r_1} + {d_2}{r_2}}&{{b_1}{r_1} + \left( {1 - {d_2} - {b_2}} \right){r_2} + {d_3}{r_3}}& \cdots &{\left( {1 - {d_N}} \right){r_N}}
	\end{array}} \right]
	$$
	با استفاده از تعریف
	
	داریم:
	$$
	{P^{n + 1}} = \left[ {\begin{array}{*{20}{c}}
			{{r_0} + d{r_1}}&{\left( {1 - d - b} \right){r_1} + 2d{r_2}}&{b{r_1} + \left( {1 - 2d - 2b} \right){r_2} + 3d{r_3}}& \cdots &{\left( {1 - Nd} \right){r_N}}
	\end{array}} \right]
	$$
	حال طبق بردار بالا،‌ برای ایندکس دلخواه
	$j$
	که
	$j = 1, \ldots ,N - 1$
	ضریب
	${r_j}$
	در عبارت
	$\mu \left( n \right)$
	برابر $j$ و در عبارت
	$\mu \left( {n + 1} \right)$
	متاثر از دو همسایه کناری، برابر مقدار زیر می‌باشد:
	
	$$
	\left( {j - 1} \right)\left( {jd} \right) + j\left( {1 - jd - jb} \right) + \left( {j + 1} \right)\left( {jb} \right) = j - jd + jb = \left( {1 - d + b} \right)j
	$$
	
	که در تمام آن‌ها عبارت
	$\left( {1 - d + b} \right)$
	در مقادیر قبلی ضرب شده است.\\
	
	در هردو عبارت ضریب ایندکس 0 برابر 0 می‌باشد ولی برای ایندکس N تفاوت برابر مقدار زیر است:
	$$
	\left( {\left( {N - 1} \right)Nd{r_N} + N\left( {1 - Nd} \right){r_N}} \right) - \left( {N{r_N}} \right) =  - Nd{r_N}
	$$
	با جمع‌بندی تمام موارد بالا می‌توان به رابطه بازگشتی مطرح شده در سوال رسید.
	
	\item
	
	باتوجه به اینکه تک تک متغیرهای عبارت
	$dN{p_N}\left( n \right)$
	غیرمنفی می‌باشد،‌ داریم:
	$$
	\mu \left( {n + 1} \right) \le (1 + b - d)\mu \left( n \right)
	$$
	
	حال اگر کران بازگشتی بالا را بسط دهیم به نامساوی زیر می‌رسیم:
	$$
	\mu \left( {n + 1} \right) \le {(1 + b - d)^n}\mu \left( 0 \right)
	$$
	
	حال اگر فرض کنیم که
	$b<d$
	می‌باشد عبارت
	${(1 + b - d)^n}$
	با میل دادن $n$ به بی‌نهایت مقدار صفر به خود می‌گیرد و کران بالای
	$\mu \left( {n + 1} \right)$
	که خود نامنفی است،‌ برابر صفر می‌شود و بنابراین حد
	$\mu \left( {n + 1} \right)$
	در حالت میل دادن $n$ به سمت بینهایت برابر صفر می‌شود.
\end{enumerate}
