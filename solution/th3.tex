\Large \textbf{درست نادرست}
\large \textbf{(۱۰ نمره)}

\normalsize \vspace{0.5cm}

\begin{enumerate}[label=(\alph*)]
	\item
	راه حل سریع:\\
	
	باتوجه به اینکه تنها یک گره جذب کننده داریم و فرایند irreducible می‌باشد پس احتمال رسیدن به گره $N$ از تمام گره ها برای 1 می‌باشد.
	راه حل طولانی:
	رابطه احتمال مطلوب را برای گره 0 می‌نویسیم:
	$$
	\begin{array}{c}
		{a_0} = p{a_1} + q{a_0}\\
		p{a_0} = p{a_1}\\
		{a_0} = {a_1}
	\end{array}
	$$
	
	با نتیجه بدست آمده اگر رابطه احتمال را برای گره 1 بنویسیم داریم:
	$$
	\begin{array}{c}
		{a_1} = q{a_0} + p{a_2}\\
		{a_1} - q{a_0} = p{a_2}\\
		p{a_1} = p{a_2}\\
		{a_1} = {a_2}
	\end{array}
	$$
	
	این روند همینطور ادامه دارد تا اینکه به گره ما قبل آخر برسیم:
	$$
	\begin{array}{c}
		{a_{N - 1}} = q{a_{N - 2}} + p\\
		p{a_{N - 1}} = p\\
		{a_{N - 1}} = 1
	\end{array}
	$$
	
	بنابراین مقدار احتمال برای تمام گره‌ها 1 می‌باشد.
	
	\item
	
	با شروع از گره 0 روابط مطلوب را مینویسیم:
	$$
	\begin{array}{c}
		{\tau _0} = p(1 + {\tau _1}) + q(1 + {\tau _0})\\
		p{\tau _0} = 1 + p{\tau _1}\\
		{\tau _0} = 2 + {\tau _1}
	\end{array}
	$$
	
	‌برای گره 1 داریم:
	$$
	\begin{array}{c}
		{\tau _1} = p(1 + {\tau _2}) + q(1 + {\tau _0})\\
		{\tau _1} - q{\tau _0} = 1 + p{\tau _2}\\
		{\tau _1} - 1 - 0.5{\tau _1} = 1 + 0.5{\tau _2}\\
		{\tau _1} = 4 + {\tau _2}
	\end{array}
	$$
	
	و اگر برای گره 2 نیز بنویسیم:
	$$
	\begin{array}{c}
		{\tau _2} = p(1 + {\tau _3}) + q(1 + {\tau _1})\\
		{\tau _2} - q{\tau _1} = 1 + p{\tau _3}\\
		{\tau _2} - 2 - 0.5{\tau _2} = 1 + 0.5{\tau _3}\\
		{\tau _2} = 6 + {\tau _3}
	\end{array}
	$$
	
	مشاهده می‌شود که برای گره‌های 0 تا
	$N-2$
	می‌توان یک رابطه کلی برای
	$i \in \left\{ {1,2, \ldots ,N - 1} \right\}$
	به صورت زیر نوشت که:
	$$
	{\tau _{i - 1}} = 2i + {\tau _i}
	$$
	حال برای گره
	$N-1$
	نیز داریم:
	$$
	\begin{array}{c}
		{\tau _{N - 1}} = p(1 + {\tau _N}) + q(1 + {\tau _{N - 2}})\\
		{\tau _{N - 1}} = p + q(1 + {\tau _{N - 2}})\\
		{\tau _{N - 1}} = 1 + 0.5{\tau _{N - 2}}
	\end{array}
	$$
	
	از طرفی در رابطه‌های بازگشتی نیز داشتیم:
	$$
	{\tau _{N - 2}} = 2N - 2 + {\tau _{N - 1}}
	$$
	
	با حل دو معادله دو مجهولی بالا برای
	${\tau _{N - 1}}$
	داریم:
	${\tau _{N - 1}} = 2N$
	با استفاده از روابط بازگشتی نیز بقیه مقادیر یافت می‌شوند.
\end{enumerate}
