\Large \textbf{دینامیک‌های پخش بیماری}
\large \textbf{(10 نمره)}

\normalsize \vspace{0.5cm}
در این سوال قصد بررسی دو مدل 
$SIR$ و $SIS$ را داریم. برای راحتی به جای بررسی جمعیت گروه‌های مستعد، ناقل و بهبود یافته، از نسبت جمعیت آن‌ها به جمعیت کل استفاده کنید، یا به عبارتی جمعیت کل را برابر 1 در نظر بگیرید.
\subsection*{الف)}
در مدل 
$SIR$ 
با تبدیل 
$\frac{dS}{dt}$ به
$\frac{dS}{dR}$ ، روابط تغییرات جمعیت دو گروه مستعد و بیمار را برحسب زمان بیابید. آیا با معادلات به دست آمده می‌توانید سرنوشت همه‌گیری را پیشبینی کنید؟ 
\subsection*{ب)}
با انتخاب دو مقدار دلخواه 
$\beta$ و $\gamma$ نشان دهید معادلات به دست آمده در بخش قبل با حل عددی سازگاری دارند.

\subsection*{پ)}
در مدل 
$SIR$ 
با 
$I_0=0.1$ (کسر افراد بیمار در لحظه اول) و 
$\gamma=0.2$، 
$\beta$ را به گونه‌ای تغییر دهید که مقدار 
$\frac{\beta}{\gamma}$ در بازه 
$(0.5,5)$ قرار بگیرد. حال به ازای هر 
$\beta$ کسر افراد بهبود یافته در پایان دینامیک (
$R_{\infty}$) را یافته و نمودار 
$R_{\infty}$ برحسب 
$\frac{\beta}{\gamma}$ را رسم کنید. نمودار حاصل را به صورت خلاصه تفسیر کنید.
\subsection*{ت)}
در مدل 
$SIS$
به ازای مقادیر متفاوت 
$I_0$ و مقادیر دلخواه
$\beta$ و $\gamma$، سرنوشت دینامیک را با استفاده از شبیه‌سازی بررسی کنید. از 
نتایج به دست آمده چه نتیجه‌ای می‌گیرید؟
آیا می‌توان با استفاده از معادلات هم به همین نتایج رسید؟
